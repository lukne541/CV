\documentclass[a4paper,10pt]{article}
% Compile with xelatex..
%A Few Useful Packages
\usepackage{marvosym}
\usepackage{fontawesome} 					%for loading fonts
\usepackage{xunicode,xltxtra,url,parskip} 	%other packages for formatting
\RequirePackage{color,graphicx}
\usepackage[usenames,dvipsnames]{xcolor}
\usepackage[big]{layaureo} 				%better formatting of the A4 page
% an alternative to Layaureo can be ** \usepackage{fullpage} **
\usepackage{supertabular} 				%for Grades
\usepackage{titlesec}					%custom \section

%Setup hyperref package, and colours for links
\usepackage{hyperref}
\definecolor{linkcolour}{rgb}{0,0.2,0.6}
\hypersetup{colorlinks,breaklinks,urlcolor=linkcolour, linkcolor=linkcolour}

%FONTS
\defaultfontfeatures{Mapping=tex-text}
%\setmainfont[SmallCapsFont = Fontin SmallCaps]{Fontin}
%%% modified for Karol Kozioł for ShareLaTeX use
\setmainfont[
SmallCapsFont = Fontin-SmallCaps.otf,
BoldFont = Fontin-Bold.otf,
ItalicFont = Fontin-Italic.otf
]
{Fontin.otf}
%%%

%CV Sections inspired by: 
%http://stefano.italians.nl/archives/26
\titleformat{\section}{\Large\scshape\raggedright}{}{0em}{}[\titlerule]
\titlespacing{\section}{0pt}{3pt}{3pt}
%Tweak a bit the top margin
%\addtolength{\voffset}{-1.3cm}

%Italian hyphenation for the word: ''corporations''
\hyphenation{im-pre-se}

%-------------WATERMARK TEST [**not part of a CV**]---------------
\usepackage[absolute]{textpos}

\setlength{\TPHorizModule}{30mm}
\setlength{\TPVertModule}{\TPHorizModule}
\textblockorigin{2mm}{0.65\paperheight}
\setlength{\parindent}{0pt}

%--------------------BEGIN DOCUMENT----------------------
\begin{document}

%WATERMARK TEST [**not part of a CV**]---------------
%\font\wm=''Baskerville:color=787878'' at 8pt
%\font\wmweb=''Baskerville:color=FF1493'' at 8pt
%{\wm 
%	\begin{textblock}{1}(0,0)
%		\rotatebox{-90}{\parbox{500mm}{
%			Typeset by Alessandro Plasmati with \XeTeX\  \today\ for 
%			{\wmweb \href{http://www.aleplasmati.comuv.com}{aleplasmati.comuv.com}}
%		}
%	}
%	\end{textblock}
%}

\pagestyle{empty} % non-numbered pages

\font\fb="[cmr10]" %for use with \LaTeX command

%--------------------TITLE-------------
\par{\centering
		{\Huge Lukas \textsc{Nee}
	}\bigskip\par}

%--------------------SECTIONS-----------------------------------
%Section: Personal Data
\section{Personal Data}

\begin{tabular}{rl}
    \textsc{Place and Date of Birth:} & Stockholm, Sweden  | 03 October 1997 \\
    \textsc{Address:}   & Björnkärrsgatan 15C, 583 36, Linköping, Sverige \\
    \textsc{Phone:}     & +46 72 3305845\\
    \textsc{email:}     & \href{mailto:lukas@nees.se}{lukas@nees.se}
\end{tabular}

%Section: Work Experience at the top
\section{Work Experience}
\begin{tabular}{r|p{11cm}}
 \emph{Current} & Part Time Junior Lab assistant at \textsc{Linköpings University}, Linköping \\\textsc{Augusti 2019}&
          \emph{Course: Perspectives in Computer and Software Technology}\\&\footnotesize{Junior assitant for 8 project groups of 3, helping them with python code and project related matters.}\\\multicolumn{2}{c}{} \\
 \textsc{Jun-Jul 2018} & Staff member at \textsc{FemtioFemPlus}, Stockholm \\&\emph{Household services at client's houses.}\\\multicolumn{2}{c}{} \\
\textsc{Mars-Aug 2017} & Consultant at \textsc{Thalamus}, \emph{Assembler at FLIR Systems Täby}\\&\footnotesize{Worked with one coworker in the warehouse to FLIR Systems then biggest assembly line in Europe. 
                                                                Responsibilities by selection include maintaining the assembly line with parts, ordering parts to maintain them in stock, book-keeping certain parts in the warehouse software (SAP), and keeping the warehouse tidy.}
\end{tabular}
%Section: Education
\section{Education}
\begin{tabular}{rl}	
 \textsc{August} 2017-2022 & Master of Science in \textsc{Computer Science and Software Engineering} \\& \textbf{Linköping University}, Linköping\\
& \textbf{Bachelor}(ongoing, ETA June-2020): ``Digital twin''. \\
&\footnotesize Part of a student team which helps IMT, LIU develope the visualization of a digital twin. \\&\footnotesize The digital twin describes bodily functions such as heart, brain, liver, fat cells,\\&\footnotesize  using inteconnected mathematical models. \hyperlink{grds_liu}{\hfill| \footnotesize Detailed List of Courses} \\\\
\textsc{August} 2013-2016& \textsc{Technology} 3 Years \\& \normalsize\textbf{Åva Gymnasium}, Stockholm\\
&\normalsize \textsc{Grade}: 18.49\\&\\

\end{tabular}

%Section: Scholarships and additional info
\section{Noteworthy Projects}
\begin{tabular}{r|p{11cm}}
  \textsc{Summer 2019} & \small Simple media HTTP webserver written in Go using REST naming convention,\\\textsc{Leisure} & \small with a sqlite3 database which is set up with Python and queried from Go. Frontend is regular JavaScript. \\\multicolumn{2}{c}{} \\
  \textsc{Oct-Dec 2019} & \small Built a StarCraft 2 bot powered by artificial intelligence together with 4 other \\\textsc{School} & \small students. The project was written in Python. My responsibility was to make the bot take sound strategic decisions. This was achieved using Supervised-learning, Bayesian Networks, and data gathered from StarCraft 2 replays.\\\multicolumn{2}{c}{} \\
 \end{tabular}


%Section: Languages
\section{Languages}
\begin{tabular}{rl}
 \textsc{Swedish:}&Mothertongue\\
\textsc{English:}&Fluent\\
\textsc{German:}&Basic Knowledge\\
\end{tabular}

\section{Computer Skills}
\begin{tabular}{rl}
 Basic Knowledge:& \textsc{C}/\textsc{C++}, \textsc{sqlite}, JavaScript, \textsc{Linux}, ubuntu, Java, Go, {\fb \LaTeX}\setmainfont[SmallCapsFont=Fontin-SmallCaps.otf]{Fontin.otf}\\
Intermediate Knowledge:& Python \\
\end{tabular}

\section{Interests and Activities}
Technology, Open-Source, Programming\\
Travelling

\newpage
\par{\centering\Large \hypertarget{grds_liu}{Master of Science in \textsc{Computer Science and Software Engineering}}\par}\large{\centering Grades\par}\normalsize
\begin{center}
\begin{tabular}{lcc}
\multicolumn{1}{c}{\textsc{Course}}&\textsc{Grade}&\textsc{Credits}\\ \hline
Discrete Mathematics	&3&	6\\
Professionalism for Engineers, part 1	&5&	1\\
Functional and Imperative Programming	&4&	11\\
Perspectives to Computer and Software Technology	&&	6\\
Object Oriented Programming and Java && 3\\
Project: Mobile and Social Applications	&4&	11\\
Professionalism for Engineers, part 4	&5	&1\\
Computer Hardware and Architecture Y	&4&	6\\
Formal Languages and Automata Theory	&3&	6\\
Linear Algebra	&3&	8\\
Software Engineering Theory	&3&	4\\ \\
		
Professionalism for Engineers, part 3	&5&	1\\
Data Structures, Algorithms and Programming Paradigms	&3&	11\\
Introductory Course in Calculus	&5&	6\\
Calculus in One Variable 1	&3&	6\\

Concurrent Programming and Operating Systems	&3&	6\\
Large-Scale Distributed Systems and Networks	&4&	8\\
Probability and Statistics	&3&	6\\
Artificial Intelligence &4& 6\\
Artificial Intelligence - Project && 6\\


\\
		
%Bachelor thesis	&	&15\\
		
    & Total&119\\\cline{2-3}
\end{tabular}
\end{center}
\end{document}
